\documentclass[12pt]{amsart}
\usepackage{amsmath}
\usepackage{amsthm}
\usepackage{amsfonts}
\usepackage{amssymb}
\usepackage[margin=1in]{geometry}
\usepackage{hyperref}
\hypersetup{
    colorlinks=true,
    linkcolor=blue
}

\theoremstyle{definition}
\newtheorem{theorem}{Theorem}[section]
\newtheorem{lemma}[theorem]{Lemma}
\newtheorem{definition}[theorem]{Definition}
\newtheorem{corollary}[theorem]{Corollary}
\newtheorem{proposition}[theorem]{Proposition}
\newtheorem{conjecture}[theorem]{Conjecture}
\newtheorem{remark}[theorem]{Remark}
\newtheorem{example}[theorem]{Example}
\newtheorem{problem}[theorem]{Problem}
\newtheorem{notation}[theorem]{Notation}
\newtheorem{question}[theorem]{Question}
\newtheorem{caution}[theorem]{Caution}

\begin{document}

\title{Homework 2: Due Sep 6}

\maketitle

For this week, please answer the following questions from the text. 
I've copied the problem itself below and the question numbers for 
your convenience. 

\begin{enumerate}
	\item (1.6) Let $a,b,c \in \mathbb{Z}$. Use the definition of divisibility to directly prove
		the following properties of divisibility.
		\begin{enumerate}
			\item If $a \mid b$ and $b \mid c$, then $a \mid c$. 
			\item If $a \mid b$ and $b \mid a$, then $a = \pm b$.
			\item If $a \mid b$ and $a \mid c$, then $a \mid (b+c)$ and $a \mid (b-c)$.
		\end{enumerate}
	\item (1.9.a) Use the Euclidean algorithm to compute $\operatorname{gcd}(291,252)$ by hand. 
	\item (1.10.a) Use the extended Euclidean algorithm to find integers $u,v$ such that 
		\begin{displaymath}
			291u + 252v = \operatorname{gcd}(291,252)
		\end{displaymath}
	\item (1.11.a-b) Let $a$ and $b$ be positive integers. 
		\begin{enumerate}
			\item Suppose that there are integers $u$ and $v$ satisfying $au + bv = 1$. Prove 
			that $\operatorname{gcd}(a,b) = 1$. 
			\item Suppose that there are integers $u$ and $v$ satisfying $au + bv = 6$. Is 
				it necessarily true that $\operatorname{gcd}(a,b) = 6$? If not, give a 
				specific counterexample, and describe in general all the possible values 
				of $\operatorname{gcd}(a,b)$. 
		\end{enumerate}
	\item (1.15) Let $m \geq 1$ be an integer and suppose that 
		\begin{displaymath}
			a_1 \equiv a_2 \mod m \ \operatorname{and} \ b_1 \equiv b_2 \mod m.
		\end{displaymath}
		Prove that 
		\begin{displaymath}
			a_1 \pm b_1 \equiv a_2 \pm b_2 \mod m \ \operatorname{and} \ a_1 \cdot b_1 \equiv 
			a_2 \cdot b_2 \mod m. 
		\end{displaymath}
	\item (1.16.a-c) Write out the following tables for $\mathbb{Z}/m\mathbb{Z}$ and 
		$(\mathbb{Z}/m\mathbb{Z})^\ast$ as done in Fig. 1.4 and 1.5 in the textbook. 
		\begin{enumerate}
			\item Make addition and multiplication tables for $\mathbb{Z}/3\mathbb{Z}$. 
			\item Make addition and multiplication tables for $\mathbb{Z}/6\mathbb{Z}$.
			\item Make a multiplication table for the unit group $(\mathbb{Z}/9\mathbb{Z})^\ast$. 
		\end{enumerate}
			
\end{enumerate}

\end{document}
