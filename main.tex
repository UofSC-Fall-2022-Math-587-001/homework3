\documentclass[12pt]{amsart}
\usepackage{amsmath}
\usepackage{amsthm}
\usepackage{amsfonts}
\usepackage{amssymb}
\usepackage[margin=1in]{geometry}
\usepackage{hyperref}
\hypersetup{
    colorlinks=true,
    linkcolor=blue
}

\theoremstyle{definition}
\newtheorem{theorem}{Theorem}[section]
\newtheorem{lemma}[theorem]{Lemma}
\newtheorem{definition}[theorem]{Definition}
\newtheorem{corollary}[theorem]{Corollary}
\newtheorem{proposition}[theorem]{Proposition}
\newtheorem{conjecture}[theorem]{Conjecture}
\newtheorem{remark}[theorem]{Remark}
\newtheorem{example}[theorem]{Example}
\newtheorem{problem}[theorem]{Problem}
\newtheorem{notation}[theorem]{Notation}
\newtheorem{question}[theorem]{Question}
\newtheorem{caution}[theorem]{Caution}

\begin{document}

\title{Homework 3}

\maketitle

For this week, please answer the following questions from the text. 
I've copied the problem itself below and the question numbers for 
your convenience. 

\begin{enumerate}
	\item (1.19) Suppose that $g^a \equiv 1 \mod m$ and $g^b \equiv 1 \mod m$. Prove that 
	\begin{displaymath}
		g^{\operatorname{gcd}(a,b)} \equiv 1 \mod m
	\end{displaymath}

	\item (1.27) Consider the congruence 
	\begin{displaymath}
		ax \equiv c \mod m
	\end{displaymath}
	\begin{enumerate}
		\item Prove that there is a solution if and only if $\operatorname{gcd}(a,m)$ divides $c$. 
		\item If there is a solution, prove that there are exactly $\operatorname{gcd}(a,m)$ 
			distinct solutions. 
	\end{enumerate}
	(Hint: Use the extended Euclidean algorithm.)

	\item (1.28) Let $\lbrace p_1,\ldots,p_r \rbrace$ be a set of prime numbers and let 
	\begin{displaymath}
		N = p_1 \cdots p_r + 1 
	\end{displaymath}
	Prove that $N$ is not divisible by any of the $p_i$. Use this fact to conclude there are infinitely 
	many prime numbers.

	\item (1.32) For each of the following primes $p$ and numbers $a$, compute $a^{-1} \mod p$ in two 
		ways: (i) Use the extended Euclidean algorithm. (ii) Use the fast power algorithm and 
		Fermat's little theorem. 
	\begin{enumerate}
		\item $p=47$ and $a=11$
		\item $p=587$ and $a=345$
		\item $p=104801$ and $a=78467$
	\end{enumerate}

	\item (1.36) This exercise begins the study of squares and square roots modulo $p$. 
	\begin{enumerate}
		\item Let $p$ be an odd prime number and let $b$ be an integer with $p \not \mid b$. Prove 
			that either $b$ has two square roots modulo $p$ or else $b$ has no square roots 
			modulo $p$. In other words, prove that the congruence 
		\begin{displaymath}
			X^2 \equiv b \mod p
		\end{displaymath}
		has either two solutions or no solutions in $\mathbb{Z}/p\mathbb{Z}$. (What happens for $p=2$? 
		What happens if $p \mid b$?)
	
		\item For each the following values of $p$ and $b$, find all the square roots of $b$ 
			modulo $p$. 
		\begin{enumerate}
			\item $(p,b) = (7,2)$
			\item $(p,b) = (11,5)$
			\item $(p,b) = (11,7)$
			\item $(p,b) = (37,3)$
		\end{enumerate}
		
		\item How many square roots does $29$ have modulo $39$? Why doesn't this contradict '
			assertion in (a)? 

		\item Let $p$ be an odd prime and let $g$ be a primitive root modulo $p$. Then any 
			number $a$ is equal to some power $g$ modulo $p$, say $a \equiv g^k \mod p$. 
			Prove that $a$ has a square root modulo $p$ if and only if $k$ is even. 
	\end{enumerate}
		
\end{enumerate}

\end{document}
